\documentclass[a4paper,10pt]{article}
\usepackage[utf8]{inputenc}
\usepackage[margin=1.0cm]{geometry}
\usepackage[table,xcdraw]{xcolor}
\usepackage{tabularx}
\usepackage{multirow}
\usepackage{graphicx}

\definecolor{headergray}{HTML}{EFEFEF}

\begin{document}
	
	\centering
	\scriptsize 
	\renewcommand{\arraystretch}{1.3} 
	
	% --- BLOQUE 1: ENCABEZADO ---
	\begin{tabularx}{\textwidth}{|X|c|}
		\hline
		\centering \rule{0pt}{3ex} \normalsize \textbf{FICHA PREDIAL} \vspace{1ex} & 
		\includegraphics[width=2.2cm]{/media/interval/Avaluos/2025/Contrato Consultoria 259/Formatos/Empresa.jpg} \\ \hline
	\end{tabularx}
	
	\vspace{-1.2pt} 
	
	% --- BLOQUE 2: INFORMACIÓN ---
	% Ajustamos el ancho a 5.0cm para que "MATRÍCULA INMOBILIARIA" entre perfecto
	\begin{tabularx}{\textwidth}{|p{5.0cm}|X|l|l|l|l|}
		\hline
		\multicolumn{6}{|c|}{\cellcolor{headergray}\textbf{1. INFORMACIÓN BÁSICA}} \\ \hline
		\textbf{PROYECTO} & \multicolumn{5}{l|}{CONTRATO DE CONSULTORIA No. 259 DE 2025, ICCU
			} \\ \hline
		\textbf{TIPO DE INMUEBLE:} & \multicolumn{5}{l|} \\ \hline
		\textbf{DIRECCIÓN:} & \multicolumn{5}{l|} \\ \hline
		\textbf{COORDENADAS:} & \textbf{LATITUD:} & %LATITUD% & \textbf{LONGITUD:} & \multicolumn{2}{l|} \\ \hline
		\textbf{VEREDA} & \multicolumn{5}{l|} \\ \hline
		\textbf{MUNICIPIO:} & %MUNICIPIO% & & \textbf{DEPARTAMENTO:} & \multicolumn{2}{l|} \\ \hline
		
		\multicolumn{6}{|c|}{\cellcolor{headergray}\textbf{2. INFORMACIÓN SOBRE TITULACIÓN}} \\ \hline
		\textbf{PROPIETARIO:} & \multicolumn{5}{l|} \\ \hline
		\textbf{TITULO DE ADQUISICIÓN:} & \multicolumn{5}{l|} \\ \hline
		\textbf{CÉDULA CATASTRAL:} & \multicolumn{5}{l|} \\ \hline
		\textbf{NUPRE:} & \multicolumn{5}{l|} \\ \hline
		% Aquí la etiqueta ya no se cortará
		\textbf{MATRÍCULA INMOBILIARIA:} & \multicolumn{5}{l|} \\ \hline
		\textbf{OBSERVACIONES JURÍDICAS:} & \multicolumn{5}{l|} \\ \hline
		\textbf{LINDEROS} & \multicolumn{5}{p{13.0cm}|}{%LINDERO_NORTE%, %LINDERO_SUR%, %LINDERO_ORIENTAL%, %LINDERO_OCCIDENTAL%} \\ \hline
		\textbf{NOTA} & \multicolumn{5}{l|}{Tomados de la consulta VUR} \\ \hline
		
		\multicolumn{6}{|c|}{\cellcolor{headergray}\textbf{3. CARACTERÍSTICAS DEL TERRENO}} \\ \hline
		\textbf{Topografía:} & %TOPOGRAFIA% & \textbf{F. Geométrica:} & IRREGULAR & \textbf{Frente:} & N/A \\ \hline
		\textbf{Fondo:} & N/A & \textbf{Relación F/F:} & N/A & \multicolumn{2}{l|}{} \\ \hline
		\textbf{NORMATIVIDAD} & \multicolumn{5}{p{13cm}|} \\ \hline
		%\textbf{PTO. REF. APROXIMADO} & \multicolumn{5}{l|}{K+0+400 A K+0+546,87} \\ \hline
		
		\multicolumn{6}{|c|}{\cellcolor{headergray}\textbf{4. CARACTERÍSTICAS DE LA CONSTRUCCIÓN}} \\ \hline
		\textbf{DESCRIPCIÓN} & \multicolumn{5}{l|}{No se requiere intervención en ninguna construcción.} \\ \hline
		\textbf{ANEXOS} & \multicolumn{5}{l|}{Cerramiento con malla eslabonada y postes, portón metálico.} \\ \hline
	\end{tabularx}
	
	\vspace{-1.2pt} 
	
	% --- BLOQUE 3: ÁREAS (Mantenemos la lógica de anchos) ---
	\begin{tabularx}{\textwidth}{|p{7cm}|p{2cm}|X|}
		\hline
		\multicolumn{3}{|c|}{\cellcolor{headergray}\textbf{5. CUADRO GENERAL DE ÁREAS}} \\ \hline
		\centering \textbf{DESCRIPCIÓN} & \centering \textbf{CANTIDAD} & \centering \textbf{OBSERVACIONES:} \tabularnewline \hline
		ÁREA TOTAL PREDIO CATASTRO ($m^2$) & \centering %AREA_CATASTRO% & \multirow{6}{=}{Existen diferencias entre las diferentes fuentes de información de área de terreno por lo cual se recomienda una aclaración de cabida y linderos. El área útil requerida se encuentra sobre la vía ya existente.} \\ \cline{1-2}
		ÁREA TOTAL PREDIO TÍTULOS ($m^2$) & \centering %AREA_DOCUMENTOS% & \\ \cline{1-2}
		ÁREA CATASTRAL REQUERIDA & \centering 599,83 & \\ \cline{1-2}
		ÁREA ÚTIL REQUERIDA & \centering 51,52 & \\ \cline{1-2}
		ÁREA TOTAL TERRENO REQUERIDA & \centering 51,52 & \\ \cline{1-2}
		ÁREA CONSTRUIDA REQUERIDA & \centering - & \\ \hline
	\end{tabularx}
	
	\vspace{-1.2pt} 
	
	% --- BLOQUE 4: LOCALIZACIÓN Y PERITO ---
	\begin{tabularx}{\textwidth}{|X|X|}
		\hline
		\multicolumn{2}{|c|}{\cellcolor{headergray}\textbf{6. LOCALIZACIÓN}} \\ \hline
		\rule{0pt}{180pt} % Altura mínima para el espacio de imagen
		\centering \includegraphics[width=0.9\linewidth] & 

	\end{tabularx}
	
	\vspace{-1.2pt}
	
	\begin{tabularx}{\textwidth}{|p{5.0cm}|X|c|}
		\hline
		\multicolumn{3}{|c|}{\cellcolor{headergray}\textbf{7. DATOS DEL ESPECIALISTA}} \\ \hline
		\textbf{ELABORÓ} & JUAN SEBASTIAN PALMA & \multirow{4}{*}{\includegraphics[width=2.2cm]{/media/interval/Avaluos/2025/Contrato Consultoria 259/Formatos/Empresa.jpg}} \\ \cline{1-2}
		\textbf{PROFESIÓN} & ING. CATASTRAL - ESP. GESTION DE PROYECTOS DE INGENIERIA & \\ \cline{1-2}
		
		\textbf{FIRMA} & \rule{0pt}{25pt} \textit{Firma Digital} & \\ \hline
	\end{tabularx}
	
\end{document}